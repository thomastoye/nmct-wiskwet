\chapter{Fysica}

\section{Kinematics in one dimension}

Powerpoint: Kinematics in one dimension

\subsection{Verschil tussen verplaatsing en afgelegde weg}

\emph{Verplaatsing} (displacement) is een vector (met een zin, richting, grootte en aangrijpingsunt).

De \emph{afgelegde weg} (travelled distance) is scalair.

\subsection{Notaties}

Een vector duiden we aan met \(\vec{x}\). Er staat niet altijd een pijltje boven, meestal gewoon een streepje erboven, bijvoorbeeld \(\bar{x}\).

De grootte van \(\vec{x}\) duiden we aan met \(\vecsize{\vec{x}}\), nu meestal gewoon men \(x\). Dit kan wat verwarrend zijn.

\( \Delta x = x_{2} - x_{1}] \)

\subsection{Gemiddelde snelheid}

In het Engels hebben we twee woorden voor snelheid: \emph{speed} en \emph{velocity}.

\[ \textrm{average velocity} = \frac{ \textrm{verplaatsing}}{ \textrm{tijdsverloop}}\]
\[ \textrm{average speed} = \frac{ \textrm{displacement}}{\textrm{tijdsverloop}} = \frac{\textrm{final position} - \textrm{initial position}}{\textrm{tijdsverloop}} \]

\emph{Speed} beschrijft hoe snel een object beweegt, \emph{velocity} geeft aan hoe snel het beweegt maar ook in welke richting (velocity is vectorieel).

\subsection{\(\frac{km}{h}\) omzetten naar \(\frac{m}{s}\)}

\[\frac{km}{h} \textrm{ naar } \frac{m}{s} \textrm{: delen door } 3.6\]
\[\frac{m}{s} \textrm{ naar } \frac{km}{h} \textrm{: vermenigvuldigen met } 3.6\]

\subsection{Opmerking over \(t\)}

Meestal spreken we af dat bij het begin van het experiment \(t = 0\). Dan moet je niet werken met \(t_1\) en \(t_2\), we laten dan gewoon de suffix weg en gebruiken \(t\).

\subsection{Ogenblikkelijke snelheid}

\[
v(t) = \lim_{\Delta \to 0}{\frac{\Delta \vec{x}}{\Delta t}} = \frac{dx}{dt} = x'(t)
\]

De laatste twee stappen (optioneel) illustreren hoe we aan de definitie van de afgeleide komen.

\subsection{Versnelling}

\[\vec{a} = \frac{\Delta \vec{v}}{\Delta t} \]

\subsection{Ogenblikkelijke versnelling}

\[ a(t) = \lim_{\Delta t \to 0}{\frac{\Delta v}{\Delta t}} = \frac{dv}{dt} \]

De uitkomst is de rico van de raaklijn aan de kromme \(v(t)\) in het punt \(P_1\).

\subsection{Ogenblikkelijke versnelling}

\[
a(t) = \frac{d}{dt} (v(t)) = \frac{d}{dt} (\frac{d}{dt} (x(t)))
\]