\chapter{Fysica}

\section{Kinematics in one dimension}

Powerpoint: Kinematics in one dimension

\subsection{Verschil tussen verplaatsing en afgelegde weg}

\emph{Verplaatsing} (displacement) is een vector (met een zin, richting, grootte en aangrijpingsunt).

De \emph{afgelegde weg} (travelled distance) is scalair.

\subsection{Notaties}

Een vector duiden we aan met \(\vec{x}\). Er staat niet altijd een pijltje boven, meestal gewoon een streepje erboven, bijvoorbeeld \(\bar{x}\).

De grootte van \(\vec{x}\) duiden we aan met \(\vecsize{\vec{x}}\), nu meestal gewoon men \(x\). Dit kan wat verwarrend zijn.

\( \Delta x = x_{2} - x_{1}] \)

\subsection{Gemiddelde snelheid}

In het Engels hebben we twee woorden voor snelheid: \emph{speed} en \emph{velocity}.

\[ \textrm{average velocity} = \frac{ \textrm{verplaatsing}}{ \textrm{tijdsverloop}}\]
\[ \textrm{average speed} = \frac{ \textrm{displacement}}{\textrm{tijdsverloop}} = \frac{\textrm{final position} - \textrm{initial position}}{\textrm{tijdsverloop}} \]

\emph{Speed} beschrijft hoe snel een object beweegt, \emph{velocity} geeft aan hoe snel het beweegt maar ook in welke richting (velocity is vectorieel).

\subsection{\(\frac{km}{h}\) omzetten naar \(\frac{m}{s}\)}

\[\frac{km}{h} \textrm{ naar } \frac{m}{s} \textrm{: delen door } 3.6\]
\[\frac{m}{s} \textrm{ naar } \frac{km}{h} \textrm{: vermenigvuldigen met } 3.6\]

\subsection{Opmerking over \(t\)}

Meestal spreken we af dat bij het begin van het experiment \(t = 0\). Dan moet je niet werken met \(t_1\) en \(t_2\), we laten dan gewoon de suffix weg en gebruiken \(t\).

\subsection{Ogenblikkelijke snelheid}

\[
v(t) = \lim_{\Delta \to 0}{\frac{\Delta \vec{x}}{\Delta t}} = \frac{dx}{dt} = x'(t)
\]

De laatste twee stappen (optioneel) illustreren hoe we aan de definitie van de afgeleide komen.

\subsection{Versnelling}

\[\vec{a} = \frac{\Delta \vec{v}}{\Delta t} \]

\subsection{Ogenblikkelijke versnelling}

\[ a(t) = \lim_{\Delta t \to 0}{\frac{\Delta v}{\Delta t}} = \frac{dv}{dt} \]

De uitkomst is de rico van de raaklijn aan de kromme \(v(t)\) in het punt \(P_1\).

\subsection{Ogenblikkelijke versnelling}

\[
a(t) = \frac{d}{dt} (v(t)) = \frac{d}{dt} (\frac{d}{dt} (x(t)))
\]

\subsection{Beweging met constante versnelling}

Door zwaartekracht: \(9.81 \frac{m}{s^2}\)

\[ v_{gem} = \frac{\Delta x}{\Delta t} = \frac{x - x_0}{t} => x = x_0 + v_{gem} * t \]

We weten dat \( v_{gem} = \frac{v_0 + v}{2} \) en \( a = \frac{\Delta v}{\Delta t} = \frac{v - v_0}{t} => v = v_0 + at \)


\[ x = x_0 + \frac{v_0 + v}{2}t
 = x_0 + \frac{1}{2} v_0 t + \frac{1}{2} v t
 = x_0 + \frac{1}{2} v_0 t + \frac{1}{2} \left(  v_0 + at \right) t = x_0 + v_0 t + \frac{1}{2} a t^2 \]
 
 \subsection{Vrije val}
 
 \subsubsection{Vanuit rust}
 
 Vanuit rust betekent dat \(v_0 = 0 \frac{m}{s}\).
 
 Gewoon invullen in formule:
 
 \[
 y(t) = y_0 + v_0t + \frac{1}{2} a t^2
 \]
 
 \(y_0\) nemen we als nul (oriëntatie van de y-as) en \(v_0\) is 0 (vanuit rust) en de versnelling is \(9.81 \frac{m}{s^2}\). Dus is de formule \( y(t) = \frac{1}{2} * 9.81 \frac{m}{s^2} * t^2  \).
 
 \subsubsection{Met beginsnelheid}
 
 Bijvoorbeeld: \( v_0 = \frac{m}{s}\)
 
 \[
 y(t) = y_0 + v_0 t + \frac{1}{2} a t^2 = 3t + 9.81 \frac{m}{s} t^2
 \]
 
 De snelheid in een punt kan je vinden door de limiet te nemen, of door af te leiden.
 
 \[ v(t) = \frac{dy}{dt} = 0 + v_{0y} + \frac{1}{2}a2t = v_{0y} + at \]
 
 \[ v(t=2s) = v_{0y} + 9.8 * 2s = 3 \frac{m}{s} + 19.60 \frac{m}{s} = 22.60 \frac{m}{s} \]
 
 \subsubsection{Bal opgooien}
 
 Typische vragen: wanneer bereikt de bal hoogste punt, duur van de beweging, snelheid bij het terugkomen, tijd wanneer de bal hoogte \(y\) bereikt.
 
 We nemen punt A als vertrekpunt van de bal, punt B als het hoogste punt, en punt C als het eindpunt (waar de bal de grond raakt).
 
 \( t_{B \to C} \) is de tijd tussen B en C.
 
 Oplossingsmethodes:
 
 \begin{enumerate}
 	\item Wanneer bereikt de bal het hoogste punt?
 	In het hoogste punt is \(v=0\). We vullen de algemene formule \(y = y_0 + v_0t + \frac{1}{2} at^2 \) in, aannemende dat \(y_0 = 0\). In het hoogste punt: \( y_{max} = v_0t - \frac{1}{2} g t^2 \). We weten dat in \(y_{max}\): \( v = 0 \frac{m}{s}\). Algemeen geldt \( a = \frac{\Delta v}{\Delta t} => v = v_0 + at \), in het hoogste punt: \( 0 = v_0\frac{m}{s} - 9.8 \frac{m}{s} \). Nu kan je \(t\) berekenen: \(t = \frac{v_0}{9.80 \frac{m}{s}}\). Dan substitueer je terug in de eerste formule om de hoogte te vinden.
 	
 	\item Snelheid bij het terugkomen
 	Algemeen: \(a = \frac{\Delta v}{\Delta t} => v = v_0 + at\). \(v_C = v_B - g * t_{B \to C} = 0 - 9.8 \frac{m}{s^2} * t_{B \to C} \).
 	
 	\(t_{B \to C}\) weet je uit de vorige stap. \(v_B\) is nul, dit is de snelheid in het hoogste punt.
 	
 	\item Tijd wanneer een bepaalde hoogte bereikt wordt
 	Algemene formule omvormen.
 \end{enumerate}
 