\chapter{Wiskunde}

\section{Machten}

\[a^n = a * a * \ldots * a\]

Met \(a \in \mathbb{R}_{0} \)

\[a^1 = a\]
\[a^0 = 1\]


Bij negatieve exponenten:
\[a^{-n} = \frac{1}{a^n} = (\frac{1}{a^n})^n \]

Bij rationele exponenten:
\[a^\frac{z}{n} = \sqrt[n]{a^z} \]
\[a^\frac{z}{n} = x \iff a^z = x^n \]

\subsection{Rekenregels}

\[a^r * a^s = a^{r+s}\]
\[\frac{a^r}{a^s} = a^{r-s}\]
\[(a^p)^q = a ^{p*q}\]
\[(a*b)^p = a^p * b^p\]

Merk op dat de laatste regel enkel geld bij een product. Een som (bijvoorbeeld \( (a + b)^p \)) moet je gaan uitwerken met het binomium van Newton.

Bij oefeningen moet je altijd proberen om de noemer wortelvrij te maken.

\[(A-B)(A+B) = A^2 - B^2\]

Tip: probeer bij oefeningen met machten wortels om te zetten naar machten, bijvoorbeeld \( \sqrt[n]{a} = a^{1/n}  \)

\subsection{Merkwaardige producten}

\[(a+b)^2 = a^2 + 2ab + b^2 \]
\[(a-b)^2 = a^2 - 2ab + b^2 \]

\[a^2-b^2 = (a-b)(a+b) \]

\[a^3 - b^3 = (a-b)(a^2 + ab + b^2)\]
\[a^3 + b^3 = (a+b)(a^2 - ab + b^2)\]

\subsection{Binomium van Newton}

Tweetermen tot de \(n\)-de macht kan je uitwerken met het binomium van Newton.

\[
(a+b)^n = \sum_{i=0}^{n}{ \Comb{i}{n} * a^i * b^{n-i} } \textrm{ met } \Comb{i}{n} = \frac{n!}{i!(n-i)!}
\]

\(\Comb{i}{n}\) is de binomiaalcoeffient.

\subsection{Driehoek van Pascal}

De driehoek van Pascal is een gemakkelijkere manier om de coefficienten te berekenen bij het binomium van Newton: de driehoek van Pascal.

\begin{tikzpicture}
\foreach \n in {0,...,7} {
	\foreach \k in {0,...,\n} {
		\node at (\k-\n/2,-\n) {$\binomialCoefficient{\n}{\k}$};
	}
}
\end{tikzpicture}

Bij een \(n\)-de macht zijn er \(n+1\) factoren in de uitkomst.

Voorbeeld:

\[
(a+b)^4
\]

Tot de vierde macht, dus er zullen vijf factoren zijn. De factoren uit de driehoek zijn 1, 4, 6, 4 en 1.

\[
(a+b)^4 = 1 a^0 b^4 + 4 a^1 b^3 + 6 a^2 b^2 + 4 a^3 b^1 + 1 a^4 b^0
\]


\section{Combinaties}

\(\Comb{i}{n}\) wil zeggen \emph{op hoeveel manieren kan je \(i\) items kiezen uit \(n\) items}.

\[
\Comb{k}{n} = \frac{n!}{k!(n-k)!}
\]

\[ \Comb{n}{n} = 1 \]

\[\Comb{1}{n} = n\]

\section{Faculteit}

\[
n! = n * (n-1) * (n-2) * \cdots * (n - (n-1))
\]

Per definitie:

\[
0! = 1\]
\[1! = 1
\]

\section{Ontbinden in factoren}

\begin{enumerate}
	\item Gemeenschappelijke factoren buiten haakjes brengen
	
	\[ 3 x^3 - x^2 + 6x - 2 = x^2(3x - 1) + 2(3x - 1) = (3x-1)(x^2 + 2) \]
	
	Merk op dat je nu zeer gemakkelijk de nulpunten kan zien.
	
	\item Toepassen van de merkwaardige producten
	
	\item Zoeken naar delers
	
	\[ 2x^3 - 3x^2 - 5x + 6 \]
	
	We kijken naar de constante term: we zoeken een deler van alle termen. Die deler moet dus ook een deler zijn van de constante term, en bij die constante term zie je snelst of een getal al dan niet een deler is.
	
	De delers van 6 zijn \( \lbrace \pm 1, \pm 2, \pm 3, \pm 6 \rbrace  \).
	
	Voor het gemak beginnen we met de kleinste deler. We beginnen nu met 1. We vullen 1 in in \(x\), wanneer we nul uitkomen, weten we dat 1 een deler is.
	
	\[ 2 * 1 - 3 * 1 - 5 + 6 = 0 \]
	
	We komen nul uit, dus we weten dat 1 een deler is. Moesten we geen nul uitkomen, ga je door met de volgende deler. Kom je nooit nul uit, dan zijn er geen delers.
	
	1 is een deler, dus kunnen we \( (x - 1) \) ( \((x-a)\) dus) afsplitsen. Zo verkrijgen we een veelterm van een graad lager.
	
	\[ (x-1)(a x^2 + bx + c) \]
	
	Met a, b, c te bepalen. Die bepaal je via het rekenschema van Horner.
	
	\[ \horner{2,-3,-5,6}{2}{2,-1,6}{2,-1,-6,0} \]
	
	De coëfficiënten zijn 2, -1 en -6.
	
	\[ 2x^3 - 3x^2 - 5x + 6 = (x-1)(2x^2 - x - 6)  \]
	
	Nu ga je verder met het ontbinden van \(2x^2 - x - 6\).
	
\end{enumerate}

\subsection{Andere methode om een tweedegraadsveelterm te ontbinden: via de discriminant}

In \(a x^2 + bx + c \) is de discriminant \( \Delta \) gegeven door \(D \textrm{ (not.)}= \Delta = b^2 - 4ac \)

Er zijn nu drie mogelijke gevallen:

\begin{enumerate}
	\item \(\Delta < 0 \): geen reële ontbinding mogelijk
	\item \(\Delta > 0 \): 2 nulpunten gegeven door \( x_{1,2} = \frac{-b \pm \sqrt{\Delta}}{2a} \)
	\item \(\Delta = 0\) 2 gelijke nulpunten gegeven door \( x_1 = x_2 = \frac{-b}{2a} \) (reden: \(\Delta = 0 => \sqrt{\Delta} = 0 \)[]
\end{enumerate}

Om nu te ontbinden, kunnen we de volgende formule gebruiken:

\[ ax^2 + bx + c = a(x - x_1)(x - x_2) \]


\section{Vergelijkingen}

\begin{enumerate}
	\item Eerste graad: \(ax + b = 0\)
	
	Met \(a \in \mathbb{R}_0, b \in \mathbb{R} \)
	
	\[=> x = \frac{-b}{a} \]
	
	\item Tweede graad: \(ax^2 + bx + c = 0\)
	
	Met \( a \in \mathbb{R}_0, b \in \mathbb{R}, c \in \mathbb{R} \)
	
	\( \Delta = b^2 - 4ac  \)
	
	\subitem \(\Delta < 0\): geen oplossingen in \(\mathbb{R}\)
	\subitem \(\Delta > 0\): twee verschillende oplossingen in \(\mathbb{R}\): \(x_{1,2} = \frac{-b \pm \sqrt{\Delta}}{2a} \)
	\subitem \(\Delta = 0\): twee gelijke oplossingen in \(\mathbb{R}\): \(x_1 = x_2 = \frac{-b}{2a}\)
	
	Indien er een ontbinding mogelijk is, dan is \( ax^2 + bx + c = 0 \) ontbindbaar tot \( a(x-x_1)(x-x_2) = 0\).
	
	Een speciaal type van \(ax^2 + bx + c = 0\) is \( af(x)^2 + bf(x) + c = 0\ \). Dit soort vergelijkingen kan je oplossen door \(t\) te introduceren, en deze gelijk te stellen aan \(f(x)\). Dan los je de vergelijking \( at^2 + bt + c = 0 \) op via de discriminant naar \(t_{1,2}\)
	
	\item Wederkerige vergelijkingen
	
	Wederkerige vergelijkingen zijn vergelijkingen met symmetrische coëfficiënten.
	
	\subitem Voorbeeld van de derde graad:
	
	\[ax^3 + bx^2 + bx + a = 0\]
	\[\iff a(x^3 + 1) + bx(x + 1) = 0\]
	\[\iff a(x+1)(x^2 - x + 1) + bx(x + 1) = 0 \]
	\[\iff (x+1)(ax^2 + (b-a)x + a) = 0 \]
	
	Daarna los je de tweedegraadsvergelijking verder op met de discriminant.
	
	\subitem Voorbeeld van de vierde graad
	
	\[ ax^4 + bx^3 + cx^2 + bx + a = 0 \]
	
	Om te beginnen, delen we door \(x^2\).	Dit mag, want de constante term is niet nul (wat betekent dat \(x=0\) geen oplossing is, wat betekent dat \(x^2\) niet nul is).
	
	\[ \iff \frac{ax^4 + bx^3 + cx^2 + bx + a}{x^2} = \frac{0}{x^2} \]
	\[ ax^2 + bx + cx^2 + \frac{b}{x} + \frac{a}{x^2} = 0 \]
	\[ a(x^2 + \frac{1}{x^2}) + b(x+\frac{1}{x}) + c = 0 \]
	
	Stel \( x + \frac{1}{x} \) gelijk aan \(t\)  \( => (x+\frac{1}{x})^2 = t^2  \iff x^2 + 2 x \frac{1}{x} + (\frac{1}{2})^2 = t^2 \iff x^2 + \frac{1}{x^2} = t^2 - 2 \)
	
	Na substitutie|
	
	\[ a(t^2 - 2) + bt + c = 0 \]
	\[ \iff at^2 + bt + (c - 2a) = 0 \]
	
	Nu bereken je de discriminant, daarna substitueer je terug naar x.
	
	\item Andere hogere-graadsvergelijkingen: via Horner

\end{enumerate}

\subsection{Gebroken vergelijkingen}

Voorbeeld:

\[ \frac{1}{x^2 - 4x} + \frac{2}{-x^2 + 16} = \frac{-1}{5x} \]

Oplossingsmethode: herinner dat \( \frac{T(X)}{N(X)} = 0 \iff T(X) = 0,  N(X) \ne 0 \). Probeer alles op gelijke noemer te zetten, zoek de nulpunten van teller en noemer. De oplossingen zijn de oplossingen van de teller zonder de oplossingen van de noemer.

\subsection{Ongelijkheden}

Oplossingsmethode: 

\begin{enumerate}
	\item Alles aan een kant brengen
	\item Nulpunten berekenen
	\item Tekenverloop bepalen
	\item Bepalen wanneer gelijkheid waar is
\end{enumerate}

\subsubsection{Gebroken ongelijkheid}

Voorbeeld:

\[ \frac{x^3 - 7x + 6}{x^2 + x} > 0 \]

\begin{enumerate}
	\item Herleiden op nul
	\item \( \frac{T(X)}{N(X)} > 0 \)
	\item Tekenverloop \(T(X)\)
	\item Tekenverloop \(N(X)\)
	\item Oplossing
\end{enumerate}
