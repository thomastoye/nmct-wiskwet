\chapter{Wiskunde}

\section{Machten}

\[a^n = a * a * \ldots * a\]

Met \(a \in \mathbb{R}_{0} \)

\[a^1 = a\]
\[a^0 = 1\]


Bij negatieve exponenten:
\[a^{-n} = \frac{1}{a^n} = (\frac{1}{a^n})^n \]

Bij rationele exponenten:
\[a^\frac{z}{n} = \sqrt[n]{a^z} \]
\[a^\frac{z}{n} = x \iff a^z = x^n \]

\subsection{Rekenregels}

\[a^r * a^s = a^{r+s}\]
\[\frac{a^r}{a^s} = a^{r-s}\]
\[(a^p)^q = a ^{p*q}\]
\[(a*b)^p = a^p * b^p\]

Merk op dat de laatste regel enkel geld bij een product. Een som (bijvoorbeeld \( (a + b)^p \)) moet je gaan uitwerken met het binomium van Newton.

Bij oefeningen moet je altijd proberen om de noemer wortelvrij te maken.

\[(A-B)(A+B) = A^2 - B^2\]

Tip: probeer bij oefeningen met machten wortels om te zetten naar machten, bijvoorbeeld \( \sqrt[n]{a} = a^{1/n}  \)